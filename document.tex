\documentclass[25pt, a0paper, portrait]{tikzposter}
\usepackage[utf8]{inputenc}
\usepackage{authblk}

\usepackage[backend=biber,style=authoryear,sorting=ynt]{biblatex}
\AtEveryBibitem{%
  \clearfield{issn} % Remove issn
  %\clearfield{doi} % Remove doi
  %\clearfield{title} % Remove title

  \ifentrytype{online}{}{% Remove url except for @online
    \clearfield{url}
  }
}
\addbibresource{document.bib}

\makeatletter
\renewcommand\maketitle{\AB@maketitle} % revert \maketitle to its old definition
%\renewcommand\AB@affilsepx{\quad\protect\Affilfont} % put affiliations into one line
\makeatother
\renewcommand\Affilfont{\large} % set font for affiliations

\title{\parbox{\linewidth}{\centering Finite temperature neutron capture study of tin isotopes}}

\author[1]{A. C. Berceanu}
\author[1, 2]{Y. F. Niu}
\author[1]{Y. Xu}

\affil[1]{ELI-NP, “Horia Hulubei” National Institute for Physics and Nuclear Engineering,
30 Reactorului Street, RO-077125, Bucharest-Magurele, Romania}
\affil[2]{School of Nuclear Science and Technology, Lanzhou University, Lanzhou 730000, China}

% todo: remove
\usepackage{blindtext}
%\usepackage{comment}

\usetheme{Board}


\begin{document}

\maketitle

\block{~}
{The r-process nucleosynthesis is responsible for the creation of about half of
the atomic nuclei heavier than iron, under extreme density and temperature
conditions. As such, the temperature dependence of neutron capture cross
sections and rates is important for determining the reaction dynamics. As shown
by the sensitivity study in~\cite{Mumpower2016}, neutron capture dynamics
on Sn isotopes are very important for r-process study.  So in this work, we
study the effect of finite temperature on the neutron-capture cross sections and
rates of even-even tin isotopes, with neutron numbers between 76 and 96.  We
compute the E1 dipole strengths, for both zero and finite temperature, using
relativistic Hartree-Bogoliubov (RHB) + quasiparticle random phase approximation
(QRPA) and finite-temperature relativistic mean-field (FTRMF) +
finite-temperature random-phase approximation (FTRPA), respectively. We use the
TALYS code for computing the corresponding cross sections, replacing its default
E1 dipole data with ours.  We find out the main effect of temperature is to
increase the low-lying E1 strength, and as a result, the neutron capture cross
sections are increased several times up to the temperature of 2 MeV.}

%\note[targetoffsetx=-9cm, targetoffsety=-6.5cm, width=0.5\linewidth]{e-mail \texttt{welcome@overleaf.com}}

\begin{columns}
    \column{0.4}
    \block{More text}{Text and more text}
 
    \column{0.6}
    \block{Something else}{Here, \blindtext}
\end{columns}
 
\begin{columns}
    \column{0.5}
    \block{A figure}
    {
        \begin{tikzfigure}
            \includegraphics[width=0.4\textwidth]{images/eli_logo.pdf}
        \end{tikzfigure}
    }
    \column{0.5}
    \block{Description of the figure}{\blindtext}
\end{columns}
 
\block{References}{
    \printbibliography[heading=none]
}

\end{document}
